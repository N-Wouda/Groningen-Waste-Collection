\documentclass[a4paper, 12pt]{article}

\usepackage[total={5.75in, 9in}]{geometry}
\usepackage{amsmath}
\usepackage{amssymb}
\usepackage{mathtools}
\usepackage{graphicx}
\usepackage{enumitem}
\usepackage{listings}
\usepackage{pythonhighlight}
\usepackage{xcolor}

\usepackage[doi=false,giveninits=true,isbn=false]{biblatex}
\addbibresource{references.bib}

\title{Notes on municipal waste}
\author{Niels Wouda}
\date{\today}

\begin{document}
    \maketitle

    \paragraph{Problem setting and current practice}
    The municipality of Groningen collects waste from underground containers.
    These containers are placed throughout the city.
    Each household is assigned to one container, which they may use to deposit their garbage bags.
    Containers have a finite capacity, and when they are full, no more garbage bags can be deposited.
    When a container is full, households have to wait before depositing the bag again.
    This is inconvenient for households, but also for the municipality, because very often garbage bags are deposited next to the full container, causing unwanted hinder.

    To deposit a garbage bag, a household must first scan a card to access the container.
    These scans are logged, giving real-time insight in the number of deposited garbage bags in the container.

    The municipality has a good idea of the capacity of an `average' garbage bag, such that they can use this scan data to estimate the currently used capacity of a container.
    This capacity is estimated based on both the `average' garbage bag, and a `volume correction factor' that is container-specific.
    The municipality combines the estimated used capacity with an estimate of the fill-rate over the past six weeks.
    This fill-rate is estimated using an OLS regression.
    Then, using the fill-rate and the currently used capacity, they estimate the moment when the container will likely be full.

    Using these forecasts, every morning the municipality prepares a plan of containers that should be emptied today.
    Although the forecasts help, this process is still highly manual: there are some containers that need to be emptied every day which the forecasts do not adequately reflect, and the volume correction factors of the containers often require tweaking as well.
    Thereafter, the planned containers are assigned to a fleet of garbage trucks, who then go out and visit the containers during the day.
    The municipality has a routing tool to do this assignment.
    In practice, the outputs of that tool result in strange assignments that often require manual tweaking (e.g., containers across a canal are assigned to the same truck, but getting there is much slower than assigning the containers to two different trucks).
    Furthermore, this route is not often followed: the garbage truck drives make their own routes, given the assigned containers.

    Summarising, there are two obvious directions for improvement.
    The first improvement direction concerns the forecasts and the selection of containers to empty today.
    The second problem concerns the efficient assignment of containers to garbage trucks.

    TODO balance number of kilometers driven against the service level of available containers.
    Variant of a (team) orienteering problem?

    \paragraph{Improving forecasting}
    TODO

    \paragraph{Assigning containers}
    TODO

    \clearpage
    \printbibliography

\end{document}
